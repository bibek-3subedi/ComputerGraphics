\documentclass[12pt]{article}

\usepackage{geometry}
\usepackage{amsmath}
\usepackage{graphicx}
\usepackage{listings}
\geometry{margin=1in}

\title{Computer Graphics Lab 5: Two Dimensional Transformations}
\author{Bibek Subedi}
\date{\today}

\begin{document}

\maketitle

\tableofcontents
\newpage

\section*{Aim}
To apply basic 2D transformations such as Translation, Scaling, Rotation, Reflection, and Shearing for a given 2D object.

\section*{Description}
In this experiment, we perform 2D transformations on a 2D object, such as a line segment. 

The 2D transformations include:

\begin{enumerate}
    \item \textbf{Translation:} Moving the object from one position to another along a straight line.
    \item \textbf{Scaling:} Changing the size of an object by scaling factors along the X and Y axes.
    \item \textbf{Rotation:} Rotating the object about the origin by an angle $\theta$.
    \item \textbf{Reflection:} Producing a mirror image of the object about a given axis.
    \item \textbf{Shearing:} Distorting the shape of the object along X or Y axis.
\end{enumerate}

\section*{Algorithms}

\subsection*{1. Translation}
\[
x' = x + t_x, \quad y' = y + t_y
\]

\subsection*{2. Scaling}
\[
x' = x \cdot s_x, \quad y' = y \cdot s_y
\]

\subsection*{3. Rotation}
\[
x' = x \cos\theta - y \sin\theta, \quad y' = x \sin\theta + y \cos\theta
\]

\subsection*{4. Reflection}
\[
\text{X-axis: } (x, -y), \quad \text{Y-axis: } (-x, y), \quad \text{Origin: } (-x, -y)
\]

\subsection*{5. Shearing}
\[
x' = x + sh_x \cdot y, \quad y' = y + sh_y \cdot x
\]

\section*{C Programs}

\subsection*{Translation Program (lab5\_1.cpp)}
\begin{lstlisting}[language=C]
#include <graphics.h>
#include <stdio.h>
#include <conio.h>

int main() {
    int gd = DETECT, gm;
    int x1, y1, x2, y2, tx, ty;

    printf("Enter x1, y1, x2, y2: ");
    scanf("%d %d %d %d", &x1, &y1, &x2, &y2);
    printf("Enter tx, ty: ");
    scanf("%d %d", &tx, &ty);

    initgraph(&gd, &gm, "");

    setcolor(WHITE);
    line(x1, y1, x2, y2);
    outtextxy(x1, y1 - 10, "Original");

    setcolor(YELLOW);
    line(x1 + tx, y1 + ty, x2 + tx, y2 + ty);
    outtextxy(x1 + tx, y1 + ty - 10, "Translated");

    getch();
    closegraph();
    return 0;
}
\end{lstlisting}

\subsection*{Scaling Program (lab5\_2.cpp)}
\begin{lstlisting}[language=C]
#include <graphics.h>
#include <stdio.h>
#include <conio.h>

int main() {
    int gd = DETECT, gm;
    int x1, y1, x2, y2;
    float sx, sy;

    printf("Enter x1, y1, x2, y2: ");
    scanf("%d %d %d %d", &x1, &y1, &x2, &y2);
    printf("Enter sx, sy: ");
    scanf("%f %f", &sx, &sy);

    initgraph(&gd, &gm, "");

    setcolor(WHITE);
    line(x1, y1, x2, y2);
    outtextxy(x1, y1 - 10, "Original");

    setcolor(GREEN);
    line(x1*sx, y1*sy, x2*sx, y2*sy);
    outtextxy(x1*sx, y1*sy - 10, "Scaled");

    getch();
    closegraph();
    return 0;
}
\end{lstlisting}

\subsection*{Rotation Program (lab5\_3.cpp)}
\begin{lstlisting}[language=C]
#include <graphics.h>
#include <stdio.h>
#include <conio.h>
#include <math.h>

int main() {
    int gd = DETECT, gm;
    int x1, y1, x2, y2;
    float theta;
    float rad;

    printf("Enter x1, y1, x2, y2: ");
    scanf("%d %d %d %d", &x1, &y1, &x2, &y2);
    printf("Enter rotation angle (degrees): ");
    scanf("%f", &theta);

    rad = theta * 3.14159 / 180;

    initgraph(&gd, &gm, "");

    setcolor(WHITE);
    line(x1, y1, x2, y2);
    outtextxy(x1, y1 - 10, "Original");

    setcolor(BLUE);
    line((int)(x1*cos(rad)-y1*sin(rad)), (int)(x1*sin(rad)+y1*cos(rad)),
         (int)(x2*cos(rad)-y2*sin(rad)), (int)(x2*sin(rad)+y2*cos(rad)));
    outtextxy((int)(x1*cos(rad)-y1*sin(rad)), (int)(x1*sin(rad)+y1*cos(rad)) - 10, "Rotated");

    getch();
    closegraph();
    return 0;
}
\end{lstlisting}

\subsection*{Reflection Program (lab5\_4.cpp)}
\begin{lstlisting}[language=C]
#include <graphics.h>
#include <stdio.h>
#include <conio.h>

int main() {
    int gd = DETECT, gm;
    int x1, y1, x2, y2;

    printf("Enter x1, y1, x2, y2: ");
    scanf("%d %d %d %d", &x1, &y1, &x2, &y2);

    initgraph(&gd, &gm, "");

    setcolor(WHITE);
    line(x1, y1, x2, y2);
    outtextxy(x1, y1 - 10, "Original");

    setcolor(MAGENTA);
    line(x1, -y1, x2, -y2); // reflection about X-axis
    outtextxy(x1, -y1 - 10, "Reflected");

    getch();
    closegraph();
    return 0;
}
\end{lstlisting}

\subsection*{Shearing Program (lab5\_5.cpp)}
\begin{lstlisting}[language=C]
#include <graphics.h>
#include <stdio.h>
#include <conio.h>

int main() {
    int gd = DETECT, gm;
    int x1, y1, x2, y2;
    float shx, shy;

    printf("Enter x1, y1, x2, y2: ");
    scanf("%d %d %d %d", &x1, &y1, &x2, &y2);
    printf("Enter shx, shy: ");
    scanf("%f %f", &shx, &shy);

    initgraph(&gd, &gm, "");

    setcolor(WHITE);
    line(x1, y1, x2, y2);
    outtextxy(x1, y1 - 10, "Original");

    setcolor(CYAN);
    line(x1 + shx*y1, y1 + shy*x1, x2 + shx*y2, y2 + shy*x2);
    outtextxy(x1 + shx*y1, y1 + shy*x1 - 10, "Sheared");

    getch();
    closegraph();
    return 0;
}
\end{lstlisting}

\section*{Results}
All programs successfully displayed the original object and its transformed version on the screen.

\section*{Conclusion}
We successfully implemented basic 2D transformations and observed the effects of Translation, Scaling, Rotation, Reflection, and Shearing on 2D objects.

\end{document}
