\documentclass{article}
\usepackage[utf8]{inputenc}
\usepackage{listings}
\usepackage{xcolor}

\definecolor{mGreen}{rgb}{0,0.6,0}
\definecolor{mGray}{rgb}{0.5,0.5,0.5}
\definecolor{mPurple}{rgb}{0.58,0,0.82}
\definecolor{backgroundColour}{rgb}{0.95,0.95,0.92}

\lstset{
    backgroundcolor=\color{backgroundColour},   
    commentstyle=\color{mGreen},
    keywordstyle=\color{magenta},
    numberstyle=\tiny\color{mGray},
    stringstyle=\color{mPurple},
    basicstyle=\footnotesize\ttfamily,
    breakatwhitespace=false,         
    breaklines=true,                 
    captionpos=b,                    
    keepspaces=true,                 
    numbers=left,                    
    numbersep=5pt,                  
    showspaces=false,                
    showstringspaces=false,
    showtabs=false,                  
    tabsize=2,
    language=C
}

\begin{document}

\section*{Experiment 4: Ellipse Generation Algorithm}

\textbf{Aim:} To implement the Ellipse Generation Algorithm for drawing an ellipse of given center $(xc, yc)$ and radius $rx$ and $ry$.

\subsection*{Description}
In an ellipse, symmetry exists between quadrants but not between the two octants of a single quadrant. Therefore, pixel positions must be calculated along the elliptical arc through one quadrant, and then positions in the remaining 3 quadrants are obtained by symmetry. The next pixel is chosen based on a decision parameter.

\subsection*{Algorithm}
\begin{enumerate}
    \item Input $rx$, $ry$, and ellipse center $(xc, yc)$, and obtain the first point on an ellipse centered on the origin as $(x_0, y_0) = (0, ry)$.
    \item Calculate the initial parameter in region 1 as $p1 = ry^2 - rx^2 ry + \frac{1}{4}rx^2$.
    \item At each $x_i$ position, starting at $i = 0$, if $p1_i < 0$, the next point along the ellipse centered on $(0, 0)$ is $(x_i + 1, y_i)$ and $p1 = p1 + 2ry^2 x_{i+1} + ry^2$. Otherwise, the next point is $(x_i + 1, y_i - 1)$ and $p1 = p1 + 2ry^2 x_{i+1} - 2rx^2 y_{i+1} + ry^2$. Continue until $2ry^2 x \ge 2rx^2 y$.
    \item $(x_0, y_0)$ is the last position calculated in region 1. Calculate the initial parameter in region 2 as $p2 = ry^2(x_0 + 0.5)^2 + rx^2(y_0 - 1)^2 - rx^2 ry^2$.
    \item At each $y_i$ position, starting at $i = 0$, if $p2_i > 0$, the next point along the ellipse centered on $(0, 0)$ is $(x_i, y_i - 1)$ and $p2 = p2 - 2rx^2 y_{i+1} + rx^2$. Otherwise, the next point is $(x_i + 1, y_i - 1)$ and $p2 = p2 + 2ry^2 x_{i+1} - 2rx^2 y_{i+1} + rx^2$.
\end{enumerate}

\subsection*{Program}
\begin{lstlisting}
#include<stdio.h>
#include<conio.h>
#include<graphics.h>
#include<math.h>

void disp();
float x, y;
int xc, yc;

void main() 
{
    int gd = DETECT, gm;
    int rx, ry;
    float p1, p2;
    clrscr();
    initgraph(&gd, &gm, "C:\\TurboC3\\BGI");

    printf("Enter the center point :");
    scanf("%d%d", &xc, &yc);
    printf("Enter the value for Rx and Ry :");
    scanf("%d%d", &rx, &ry);

    x = 0;
    y = ry;
    disp();

    /* -------- Region 1 -------- */
    p1 = (ry * ry) - (rx * rx * ry) + (rx * rx) / 4.0;
    while ((2.0 * ry * ry * x) <= (2.0 * rx * rx * y)) 
    {
        x++;
        if (p1 <= 0)
            p1 = p1 + (2.0 * ry * ry * x) + (ry * ry);
        else 
        {
            y--;
            p1 = p1 + (2.0 * ry * ry * x) - (2.0 * rx * rx * y) + (ry * ry);
        }
        disp();
    }

    /* -------- Region 2 -------- */
    p2 = (ry * ry) * (x + 0.5) * (x + 0.5) + (rx * rx) * (y - 1) * (y - 1) - (rx * rx * ry * ry);
    while (y > 0) 
    {
        y--;
        if (p2 > 0)
            p2 = p2 + (rx * rx) - (2.0 * rx * rx * y);
        else 
        {
            x++;
            p2 = p2 + (2.0 * ry * ry * x) - (2.0 * rx * rx * y) + (rx * rx);
        }
        disp();
    }
    getch();
    closegraph();
}

void disp() 
{
    delay(50);
    putpixel(xc + x, yc + y, 10);
    putpixel(xc - x, yc + y, 10);
    putpixel(xc + x, yc - y, 10);
    putpixel(xc - x, yc - y, 10);
}
\end{lstlisting}

\end{document}